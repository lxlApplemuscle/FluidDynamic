% !TeX program = xelatex
\documentclass[cn, twoside]{myModel}
\TitleName{Geophysical fluid dynamics}
\basicInformation{AppleMuscle}{lixianlongouc@gmail.com}
\backgroundName{1564367343.eps}{.7}{.1}

\begin{document}
	\maketitle
	\begin{cnabstract}
		\par{}\noindent{}\textcolor{blue!70!black}{\textbf{各种坐标系中的矢量算子}}
		\begin{itemize}
			\item
			Cartesian coordinate system(x, y, z):
			\begin{itemize}
				\item[]
				\textbf{Gradient} $\nabla\psi=\frac{\partial \psi}{\partial x}\boldsymbol{i}+\frac{\partial \psi}{\partial y}\boldsymbol{j}+\frac{\partial \psi}{\partial z}\boldsymbol{k}$ \quad
				\textbf{Divergence} $\nabla\cdot\overrightarrow{V}=\frac{\partial u}{\partial x}+\frac{\partial v}{\partial y}+\frac{\partial w}{\partial z}$\\
				\textbf{Rotation} $\nabla\times\overrightarrow{V}=(\frac{\partial w}{\partial y}-\frac{\partial v}{\partial z})\boldsymbol{i}+(\frac{\partial u}{\partial z}-\frac{\partial w}{\partial x})\boldsymbol{j}+\textcolor{blue}{(\frac{\partial v}{\partial x}-\frac{\partial u}{\partial y})\boldsymbol{k}}$
			\end{itemize}
			\item
			Cylindrical coordinate system(r, $\theta$, z):
			\begin{itemize}
				\item[]
				\textbf{Gradient} 
				$\nabla\psi=\frac{\partial \psi}{\partial r}\boldsymbol{i}+\frac{1}{r}\frac{\partial \psi}{\partial \theta}\boldsymbol{j}+\frac{\partial \psi}{\partial z}\boldsymbol{z}$ \quad
				\textbf{Divergence} $\nabla\cdot\overrightarrow{V}=\frac{1}{r}\frac{\partial}{\partial r}(ru)+\frac{1}{r}\frac{\partial u}{\partial \theta}$ \\
				\textbf{Rotation} $\overrightarrow{k}\cdot(\nabla\times\overrightarrow{V})=\frac{1}{r}\frac{\partial}{\partial r}(rv)-\frac{1}{r}\frac{\partial u}{\partial \theta}$ \quad
				$\nabla^2_{h}\psi=\frac{1}{r}\frac{\partial}{\partial r}(r\frac{\partial \psi}{\partial r})+\frac{1}{r^2}\frac{\partial^2 \psi}{\partial \theta^2}$
			\end{itemize}
			\item
			Spherical coordinate system($\lambda$, $\varphi$, r):
			\begin{itemize}
				\item[]
				\textbf{Gradient} $\nabla\psi=\frac{\boldsymbol{i}}{r\cos{\varphi}}\frac{\partial \psi}{\partial \lambda}+\frac{\boldsymbol{j}}{r}\frac{\partial \psi}{\partial \varphi}+\boldsymbol{k}\frac{\partial \psi}{\partial r}$ \quad
				\textbf{Divergence}
				$\nabla\cdot\overrightarrow{V}=\frac{1}{r\cos{\varphi}}[\frac{\partial u}{\partial \lambda}+\frac{\partial}{\partial \varphi}(v\cos{\varphi})]$ \\
				\textbf{Rotation}
				$\overrightarrow{k}\cdot(\nabla\times\overrightarrow{V})=\frac{1}{r\cos{\varphi}}[\frac{\partial v}{\partial \lambda}-\frac{\partial}{\partial \varphi}(u\cos{\varphi})]$ \quad
				$\nabla^2_{h}\psi=\frac{1}{r\cos^2{\varphi}}[\frac{\partial^2 \psi}{\partial \lambda^2}+\cos{\varphi}\frac{\partial}{\partial \varphi}(cos{\varphi}\frac{\partial psi}{\partial \varphi})]$
			\end{itemize}
		\end{itemize}
		\par{}\noindent{}\textcolor{blue!70!black}{\textbf{积分定理}}
			\normalsize
			\begin{itemize}
				\item[(1)] Divergence theorem \\
				\begin{minipage}{6cm}
					\centering
					{$\displaystyle\int_{V} \nabla\cdot\overrightarrow{A}dV=\displaystyle\int_{\sum} \overrightarrow{A}\cdot\overrightarrow{n}d\sigma$}
				\end{minipage}
				\begin{minipage}{9.5cm}
					\raggedright
					{
						式中$dV$为体积元,$V$为体积,$d\sigma$为面积元,$\sum$为包围体积$V$的曲面,$\overrightarrow{n}$为曲面$\sum$的外法线方向上的单位矢量。
					}
				\end{minipage} \\ [.5em]
				\begin{minipage}{6cm}
					\centering
					{$\displaystyle\int_{S}\nabla\cdot\overrightarrow{F}dS
					=\displaystyle\oint_{C} \overrightarrow{F}\cdot\overrightarrow{n}dl$}
				\end{minipage}
				\begin{minipage}{9.5cm}
					\raggedright
					{
						式中$\overrightarrow{F}$为二维矢量,$dS$为平面上面积元,$S$为平面面积,$C$为包围$S$的曲线,$dl$为沿着$C$的线元,$\overrightarrow{n}$为曲线$C$的外法线方向上的单位矢量。
					}
				\end{minipage}
				\item[(2)] Stokes' theorem \\
				\begin{minipage}{6cm}
					\centering
					{$\displaystyle\int_{V}\nabla\times\overrightarrow{A}dV
					 =\displaystyle\int_{\sum}\overrightarrow{n}\times\overrightarrow{A}d\sigma$}\\
					 {$\displaystyle\int_{S}\overrightarrow{k}\cdot\nabla\times\overrightarrow{F}dS
					 =\displaystyle\oint_{C}\overrightarrow{F}\cdot\overrightarrow{\tau}dl$} \\
					 {$\displaystyle\int_{V}\nabla{}adV=\displaystyle\oint_{\sum}a\overrightarrow{n}d\sigma$}
				\end{minipage}
				\begin{minipage}{9.5cm}
					\raggedright
					{
						式中$\overrightarrow{k}$是平面$S$的法线方向上的单位矢量,$\overrightarrow{\tau}$是曲线$C$的切线方向上的单位矢量,其他符号同上。
					}
				\end{minipage}
			\end{itemize}
	\end{cnabstract}
	
	\newpage
	\large
	\section{基础知识}
		\subsection{\textcolor{blue!70!black}{基本方程}}
			\begin{itemize}
				\item 
				地转($\Omega=7.3\times10^{-5}ms^{-1}$)\quad 薄层($\frac{H}{L}\ll1$)\quad  
				层化($\rho\neq{}constant$)
				\item 基本控制方程:
				\begin{itemize}
					\item[]
					动量方程$\rho(\frac{d\overrightarrow{V}}{dt}+2\overrightarrow{\Omega}\times\overrightarrow{V})=-\nabla{p}-\rho\nabla\Phi+\overrightarrow{F}$
					$\left(\overrightarrow{F}=\mu\nabla^{2}\overrightarrow{V}+\frac{\mu}{3}\nabla(\nabla\cdot\overrightarrow{V})\right)$ 
					\item[]
					连续方程$\frac{dp}{dt}+\rho\nabla\cdot\overrightarrow{V}=0$
					\item[]
					状态方程$\left\{
					\begin{minipage}{14cm}
						\raggedright
						$\mbox{海洋}\left\{
						\begin{minipage}{6cm}
							\raggedright
							{$\rho=\rho_{0}[1-\alpha_{T}(T-T_{0})+\alpha_{S}(S-S_{0})]$}\\
							{$\frac{dT}{dt}=K_{T}\nabla^{2}T+\frac{Q_{H}}{\rho{}C_{v}}$} \\
							{$\frac{dS}{dt}=K_{S}\nabla^{2}S$}
						\end{minipage}
						\right.$ 
						\begin{minipage}{6cm}
							\raggedright
							{$\alpha_{T},\alpha_{S}$为热,盐膨胀系数} \\
							{$K_{T},K_{S}$为热,盐扩散系数} \\
							{$Q_{H}$为热源或汇,$C_{v}$为比热}
						\end{minipage}
						\\ [.3em]
						$\mbox{大气}\left\{
						\begin{minipage}{6cm}
							\raggedright
							{$\rho=\frac{p}{RT}$}\\
							{$\frac{dT}{dt}=\kappa\nabla^{2}T-\frac{Q}{C_{p}}$} 
						\end{minipage}
						\right.$
						\begin{minipage}{6cm}
							\raggedright
							{R为干空气气体常数}, {$\kappa$为热扩散率}, {$C_{p}$定压比热}, 
							{$Q$为加热率}
						\end{minipage}
					\end{minipage}
					\right.$
				\end{itemize}
				\item 无粘流体:
					\begin{itemize}
						\begin{minipage}{10cm}
							\raggedright
							\item[]基本方程
								$\frac{d\overrightarrow{V}}{dt}=-\frac{V_{p}}{\rho}-2\overrightarrow{\Omega}\times\overrightarrow{v}+\overrightarrow{g}$ 
							\item[]球坐标($\lambda$, $\theta$, r)
								\begin{equation*} \normalsize
									\begin{split}
										\frac{d\overrightarrow{V}}{dt} & = \left(\frac{du}{dt}-\frac{uv\tan{\theta}}{r}+\frac{uw}{r}\right)\overrightarrow{i} \\
										&+\left(\frac{dv}{dt}-\frac{u^{2}\tan{\theta}}{r}+\frac{vw}{r}\right)\overrightarrow{j}+\left(\frac{dw}{dt}-\frac{u^{2}+v^{2}}{r}\right)\overrightarrow{k}
									\end{split}
								\end{equation*}
						\end{minipage}
						\begin{minipage}{5.5cm}
							\begin{figure}[H]
								\centering
								\includegraphics[width=.75\textwidth]{sp_system}
							\end{figure}
						\end{minipage}
						\item[] \small
							$\large{\mbox{地转角速度}}\left\{
							\begin{minipage}{15cm}
								\raggedright
								{$\overrightarrow{\Omega}=\Omega\cos{\theta}\overrightarrow{j}+\Omega\sin{\theta}\overrightarrow{k}$} \\
								{$2\overrightarrow{\Omega}\times\overrightarrow{v}=(2\Omega\omega\cos{\theta}-2\Omega{}v\sin{\theta})\overrightarrow{i}+2\Omega{}u\sin{\theta}\overrightarrow{j}-2\Omega{}u\cos{\theta}\overrightarrow{k}$}
							\end{minipage}
							\right.$ 
						\item[] 
						$\large{\mbox{分量形式为}}\left\{
						\begin{minipage}{15cm}
							\raggedright
							{$\frac{du}{dt}-(2\Omega+\frac{u}{r\cos{\theta}})(v\sin{\theta}-w\cos{\theta})=-\frac{1}{\rho{r}\cos{\theta}}\frac{\partial p}{\partial \lambda}$} \\
							{$\frac{dv}{dt}+\frac{wv}{r}+(2\Omega+\frac{u}{r\cos{\theta}})u\sin{\theta}=-\frac{1}{\rho{r}}\frac{\partial p}{\partial \theta}$} \\
							{$\frac{dw}{dt}-\frac{u^2+v^2}{r}-2\Omega{u}\cos{\theta}=-\frac{1}{\rho}\frac{\partial p}{\partial r}-g$}
						\end{minipage}
						\right.$ 
					\end{itemize}
				\item 动量方程简化为:
					$\left\{
					\begin{minipage}{6cm}
						\raggedright
						{$\frac{du}{dt}-fv-\frac{uv}{a}\tan{\theta}=-\frac{1}{\rho{a}\cos{\theta}}\frac{\partial p}{\partial \lambda}$} \\
						{$\frac{dv}{dt}+fu+\frac{u2}{a}\tan{\theta}=-\frac{1}{\rho{a}}\frac{\partial p}{\partial \theta}$} \\
						{$0=-\frac{1}{\rho}\frac{\partial p}{\partial z}-g$}
					\end{minipage}
					\right.$ 
					\begin{minipage}{8cm}
						\raggedright 其中:
						{$\frac{d}{dt}=\frac{\partial}{\partial t}+\frac{u}{a\cos{\theta}}\frac{\partial}{\partial \lambda}+\frac{v}{a}\frac{\partial}{\partial \theta}+w\frac{\partial}{\partial z}$} \\
						\qquad{$f=2\Omega\sin{\theta}$}(科氏参数)
					\end{minipage}
				\item 连续方程为:$\frac{d\rho}{dt}+\rho\left[\frac{1}{a\cos{\theta}}\frac{\partial u}{\partial \lambda}+\frac{1}{a\cos{\theta}}\frac{\partial}{\partial \theta}(v\cos{\theta})+\frac{\partial w}{\partial z}\right]=0$
			\end{itemize}

		\subsection{\textcolor{blue!70!black}{基本原理}}
			\begin{itemize}
				\item \textbf{$f$}平面近似 \\
				这是对地转参数采用的一种近似,在中纬度地区,可以将$f$作为常数处理。 \\ 
			 	纬度为$\theta_{0}(\delta{x},\delta{y},\delta{z})\approx(a\cos{\theta_{0}}\delta\lambda, a\delta\theta_{0}, \delta{z})$\quad $f_{0}=2\Omega\sin{\theta_{0}}=constant$ \\ [.5em]简化动力方程
				$\left\{
				\begin{minipage}{7cm}
					\raggedright\large
					{$\frac{\partial u}{\partial t}+(\overrightarrow{V}\cdot\nabla)u-f_{0}v=-\frac{1}{\rho}\frac{\partial p}{\partial x}$} \\
					{$\frac{\partial v}{\partial t}+(\overrightarrow{V}\cdot\nabla)v+f_{0}u=-\frac{1}{\rho}\frac{\partial p}{\partial y}$} \\
					{$\frac{\partial p}{\partial z}=-\rho{g}$}
				\end{minipage}
				\right.$ 
				\begin{minipage}{7cm}
					\raggedright\large
					{$\frac{d\overrightarrow{V}}{dt}+\overrightarrow{f_{0}}\times\overrightarrow{V}=-\frac{1}{\rho}\nabla_{z}p$} \\
					$\frac{d\overrightarrow{V}}{dt}=\frac{\partial \overrightarrow{V}}{\partial t}+(\overrightarrow{V}\cdot\nabla)\overrightarrow{V}$
				\end{minipage}
			    \item $\beta$平面近似 \\
			    流体运动的尺度足够大(或者在赤道附近),要考虑地球曲率的影响;为了即使用直角坐标系,又把$f$随纬度的变化考虑在内,这种近似为$\beta$平面近似: 
			    $$
			    f=f_{0}+\beta{}y, \beta=\frac{\partial f}{\partial y}=\frac{\partial(2\Omega\sin{\theta})}{a\partial \theta}=\frac{2\Omega\cos{\theta_{0}}}{a}
			    $$
			    \item 静力近似 $\frac{\partial p}{\partial z}=-\rho{g}$ \\
			    假定 
			    $p$(x,y,z)=$\overline{p}$(z)+$p^{\prime}$(x,y,z,t),\enspace|$p^{\prime}$|$\ll$$\overline{p}$\quad $\rho$(x,y,z)=$\overline{\rho}$(z)+$\rho^{\prime}$(x,y,z,t),\enspace|$\rho^{\prime}$|$\ll$$\overline{\rho}$ \\
			    \begin{minipage}{6cm}
			    	\raggedright
			    	\begin{equation*}
			    		\begin{split}
			    			\frac{\partial(\overline{p}+p^{\prime})}{\partial z}&=-(\overline{\rho}+\rho^{\prime})g \\
			    			\frac{\partial p^{\prime}}{\partial z}&=-\rho^{\prime}g
			    		\end{split}
			    	\end{equation*}
			    \end{minipage}
			    \begin{minipage}{8cm}
			    	\centering
			    	在静力近似中,可以使用扰动量来代替平均量,将精确度提高两个量级。
			    \end{minipage} \\ [.3em]
			    静力近似不适用的情况:
			    \begin{equation*} 
			   		\begin{split}
			   			\frac{dw}{dt}-&2\Omega{u}\cos{theta}=-\frac{1}{\rho}\frac{\partial p}{\partial z}-g \\
			   			&=-\frac{1}{\overline{\rho}(1+\frac{\rho^{\prime}}{\overline{\rho}})}\frac{\partial (\overline{p}+p^{\prime})}{\partial z}-g\approx-\frac{1}{\overline{\rho}}\frac{\partial p^{\prime}}{\partial z}-\frac{\rho^{\prime}}{\overline{\rho}}g
			   		\end{split}
			   	\end{equation*}\\
			   	由水平动量方程可得:\textcolor{violet!70!black}{$p^{\prime}\sim2\Omega\rho{UL}$}$\Rightarrow$
			   	\begin{minipage}{5cm}
			   		\raggedright\Large
			   		$\frac{2\Omega{u}\cos{\theta}}{\frac{1}{\overline{\rho}}\frac{\partial p^{\prime}}{\partial z}}$$\sim$$\frac{H}{L}\ll1$
			   	\end{minipage} \\ [.3em]	
			   	为了满足静力近似,即
			   	\begin{minipage}{3.4cm}
			   	\raggedright
			   		$\frac{dw}{dt}\ll\frac{\rho^{\prime}}{\overline{p}}g:T\sim\frac{L}{U}$
			   	\end{minipage} $\Rightarrow$
			   	\begin{minipage}{5cm}
			   	\raggedright
			   		\textcolor{blue!70!black}{$W\ll\frac{L\rho^{\prime}}{U\overline{p}}g$}
			   	\end{minipage}
			   	\item \textbf{Boussinesq}近似 \\
			   		在密度变化不大的水体中,除了浮力项中的密度外,其他项的密度可以当作常数。假定
			   		\begin{minipage}{5.cm}
			   			\begin{equation*} 
			   				\begin{split}
			   					\rho&=\rho_{0}+\partial\rho(x,y,z,t)\\
			   						&=\textcolor{darkgray}{\rho_{0}+\tilde{\rho}(z)+\rho^{\prime}(x,y,z,t)} \\
			   						&=\textcolor{darkgray}{\tilde{\rho}(z)+\rho^{\prime}(x,y,z,t)} \\
			   					p&=\overline{p}+\partial{p}(x,y,z,t)
			   				\end{split}
			   			\end{equation*}
			   		\end{minipage}
			   		\begin{minipage}{10.5cm}
			   			\centering
			   			$\rho_{0}=constant$\quad|$\partial \rho$|,|$\partial \rho$|,|$\rho^{\prime}$|$\ll$$\rho_{0}$\quad |$\partial p$|$\ll$$\tilde{p}$ \\
			   			\textcolor{blue!70!black}{海洋:$\rho_{0}\cong1000kg/m^{3}$\quad $\tilde{\rho}\cong10kg/m^3$\quad$\rho^{\prime}\cong0.1kg/m^{3}$} \\
			   			\textcolor{green!30!black}{大气:$\rho_{0}\cong0.5kg/m^{3}$\quad $\tilde{\rho}\cong0.5kg/m^3$\quad$\rho^{\prime}\cong0.005kg/m^{3}$}
			   		\end{minipage} \\
			   		带入动量方程可得:
			   		$$ 
			   		(\rho_{0}+\partial{\rho})\left(\frac{d\overrightarrow{V}}{dt}+2\Omega\times\overrightarrow{V}\right)=-\nabla\partial{p}-\frac{\partial \overline{p}}{\partial z}\overrightarrow{k}-g(\rho_{0}+\partial \rho)\overrightarrow{k}
			   		$$
			   		带入静力近似$\frac{\partial \overrightarrow{p}}{\partial z}$=-$\rho_{0}g$,方程可写为:
			   		\begin{minipage}{8cm}
			   			\centering
			   			$(\rho_{0}+\partial{\rho})\left(\frac{d\overrightarrow{V}}{dt}+2\Omega\times\overrightarrow{V}\right)=-\nabla\partial{p}-g\partial\rho\overrightarrow{k}$
			   		\end{minipage} \\
			   		由|$\partial \rho$|$\ll$$\rho_{0}$,\textcolor{red!70!white}{动量方程}进而可写为:
			   		$$
			   			\frac{d\overrightarrow{V}}{dt}+2\Omega\times\overrightarrow{V}=-\frac{\nabla\partial p}{\rho_{0}}-\frac{g\partial \rho}{\rho_{0}}\overrightarrow{k}(-\frac{g\partial p}{\rho_{0}}\mbox{为浮力})
					$$
			   		连续方程:
			   		\begin{minipage}{2cm}
			   			\raggedright
			   			$\frac{d\rho}{dt}+\rho\nabla\cdot\overrightarrow{V}$
			   		\end{minipage}\textcolor{red!50!black}{$\Rightarrow$} 
			   		\begin{minipage}{6cm}
			   		\raggedright
			   		$\frac{d\partial\rho}{dt}+(\rho_{0}+\partial \rho)\nabla\cdot\overrightarrow{V}=0$
			   		\end{minipage} \\
			   		\mybox[red]{Q:因为$\frac{d}{dt}\sim\nabla\cdot\overrightarrow{V}$,因此又可写成$\nabla\cdot\overrightarrow{V}=0$}
			   		\item \textbf{Brunt-V$\ddot{a}$is$\ddot{a}$l$\ddot{a}$}频率\\
			   			不可压缩流体(海洋):\enspace$\rho=\rho(z)$\\
			   			流体微团A由$z$升到$z+dz$,$\delta\rho$为流体微团A和它周围的密度差:
			   			$$\delta\rho=rho_{A}-\rho_{B}=-\frac{\partial \rho}{\partial z}dz$$
			   			流体微团所受到的恢复力:\enspace$F=-g\delta\rho=g\frac{\partial\rho}{\partial z}dz$,\enspace 由牛顿第二定律可得:
			   			$$\rho\frac{\partial^2 (dz)}{\partial t^2}-g\frac{\partial\rho}{\partial z}=0\quad\textcolor{red!70!black}{\Longrightarrow}\quad\frac{\partial^2(dz)}{\partial t^2}+N^{2}dz=0\enspace\left(N^2=-\frac{g}{\rho}\frac{\partial \rho}{\partial z}\right)$$
			   			\begin{itemize}
			   				\item 当$N^2>0$时,方程的通解为:\enspace$dz(t)=A\cos{Nt}+B\sin{Nt}$
			   				$$
			   				When\enspace t=0,\enspace dz(t)=0\quad\textcolor{red!70!black}{\Longrightarrow}\quad A=0\quad\textcolor{red!70!black}{\Longrightarrow}\quad dz(t)=B\sin{Nt}=z_{0}\sin{Nt}
			   				$$
			   				该解表明流体微团在平衡高度上下震荡,$z_{0}$时震荡的振幅,其频率即为Brunt-V$\ddot{a}$is$\ddot{a}$l$\ddot{a}$频率或称为浮力频率:\enspace$N=\left[-\frac{g}{\rho_{0}}\frac{\partial \rho}{\partial z}\right]^{1/2}$
			   				\item 当$N^2<0$时,方程的通解为:\enspace$dz(t)=Ae^{i\sqrt{N^2}t}+Be^{-i\sqrt{N^2}t}$ 
			   				$$
			   				When\enspace t=0,\enspace dz(t)=0\quad\textcolor{red!70!black}{\Longrightarrow}\quad A+B=0\quad\textcolor{red!70!black}{\Longrightarrow}\quad dz(t)=B\left(e^{-i\sqrt{N^2}t}-e^{i\sqrt{N^2}t}\right)
			   				$$
			   				因为$i\sqrt{N^2}$和$-i\sqrt{N^2}$是实数,该解表明流体微团离平衡高度越来越远。
			   			\end{itemize}
			   			可压缩流体(大气):\enspace$\rho=\rho(z),\enspace p=p(z)$ \\
			   			流体微团A由$z$升到$z+dz$,它的密度可写为:\enspace$\rho_A+\Delta\rho_A=\rho_A(z)+\frac{\partial \rho}{\partial p}\frac{d\rho}{dz}dz$ \\
			   			流体微团B的密度为:\enspace$\rho_{B}=\rho_A(z)+\frac{\partial \rho}{\partial z}dz$ 
			   			\begin{equation*}
			   				\begin{split}
			   					&\textcolor{blue!70!black}{
			   						\delta\rho=\rho_A+\Delta\rho_A-\rho_{B}=\frac{\partial \rho}{\partial p}\frac{dp}{dz}dz-\frac{\partial\rho}{\partial z}dz
			   					} \\
			   					\Longrightarrow-&\frac{g\delta\rho}{\rho}=-\frac{g}{\rho}\left(\frac{\partial \rho}{\partial p}\frac{dp}{dz}-\frac{\partial\rho}{\partial z}\right)dz=-N^{2}dz,\enspace N^2=\frac{g}{\rho}\left(\frac{\partial\rho}{\partial p}\frac{dp}{dz}-\frac{\partial\rho}{\partial z}\right)
			   				\end{split}
			   			\end{equation*}
			   			理想大气中$$
			   			\theta=T\left(\frac{p_0}{p}\right)^\frac{R}{C_p},\enspace\rho=\frac{p}{RT}\quad\Longrightarrow\quad\rho=\frac{p_0}{R\cdot\theta}\left(\frac{p}{p_0}\right)^{\frac{1}{\gamma}},\enspace \mbox{其中}\enspace\gamma=\frac{C_p}{C_v}, C_v=C_p-R 
			   			$$
			   			$$
			   			N^2=\frac{g}{\rho}\left(\frac{\partial \rho}{\partial p}\frac{dp}{dz}-\frac{\partial\rho}{\partial z}\right)=g\textcolor{red!70!black}{\left(\frac{1}{\gamma p}\frac{\partial p}{\partial z}-\frac{1}{\rho}\frac{\partial \rho}{\partial z}\right)}=\textcolor{blue!70!black}{\frac{g}{\theta}\left(\frac{\partial \theta}{\partial z}\right)} 
			   			$$
			   			\begin{box1}{\textbf{Prove:}}
			   				Here we prove:\textcolor{blue!70!black}{$\quad\frac{1}{\gamma p}\frac{\partial p}{\partial z}-\frac{1}{\rho}\frac{\partial \rho}{\partial z}=\frac{1}{\theta}\frac{\partial \theta}{\partial z}$} \\
			   				Set\textcolor{blue!70!black}{$\frac{\partial F}{\partial z}=F^{\prime}$} and \textcolor{blue!70!black}{$F=f_1f_2$}, then we have: \textcolor{blue!70!black}{$F^{\prime}=f_1^{\prime}f_2+f_2^{\prime}f_1$} and \textcolor{blue!70!black}{$\frac{F^\prime}{F}=\frac{f^{\prime}_1}{f_1}+\frac{f^{\prime}_2}{f_2}$}. \\
			   				For the formula \textcolor{blue!70!black}{$\rho=\frac{p_0}{R\cdot\theta}\left(\frac{p}{p_0}\right)^{\frac{1}{\gamma}}$}, set\textcolor{blue!70!black}{$F=\rho,\enspace f_1=\frac{1}{\theta},\enspace f2=\frac{p_0}{R}\left(\frac{p}{p_0}\right)^{\frac{1}{\gamma}}$}
			   				\begin{equation*}
			   					\textcolor{blue}{\begin{split}
			   							\frac{1}{rho}\frac{\partial \rho}{\partial z}=\frac{\rho^\prime}{\rho}&=\theta\cdot\left(-\frac{1}{\theta^2}\right)\theta^{\prime}+\left[\frac{p_0}{R}\left(\frac{p}{p_0}\right)^\frac{1}{\gamma}\right]^{-1}\cdot\left[\frac{p_0}{R}\left(\frac{p}{p_0}\right)^\frac{1}{\gamma}\right]^{\prime} \\
			   							&=-\frac{\theta^{\prime}}{\theta}+\left[\frac{p_0}{R}\left(\frac{p}{p_0}\right)^\frac{1}{\gamma}\right]^{-1}\cdot\frac{1}{\gamma}\frac{p_0}{R}\left(\frac{p}{p_0}\right)^{\frac{1}{\gamma}-1}\cdot\frac{1}{p_0}p^{\prime} \\
			   							&=-\frac{\theta^{\prime}}{\theta}+\frac{1}{\gamma P}p^{\prime}=-\frac{1}{\theta}\frac{\partial \theta}{\partial z}+\frac{1}{\gamma P}\frac{\partial p}{\partial z}
			   					\end{split}}
			   				\end{equation*}
			   			\end{box1}
			\item 地转平衡(\textbf{Geostrophic Balance}) \\
				Rossby Number: $R_{0}\equiv\frac{U^{2}}{L}/fU\equiv\frac{H}{fL}$,旋转时间尺度($\frac{1}{f}$)/平流时间尺度($\frac{L}{U}$),相对涡度($\frac{U}{L}$)/行星涡度(f)
				\begin{itemize}
					\item[] $R_0\ll1$,得到地转平衡方程:\enspace $\overrightarrow{f}\times\overrightarrow{V}=-\frac{1}{\rho}\nabla_{z}p,\enspace\overrightarrow{f}=2\Omega\sin{\theta}\overrightarrow{k}$ \\
					\begin{minipage}{4.5cm}
						\centering
						直角坐标系: \\
						{$fu_{g}=-\frac{1}{\rho}\frac{\partial p}{\partial y}$\enspace
						 $fv_{g}= \frac{1}{\rho}\frac{\partial p}{\partial x}$}
					\end{minipage} 
					\begin{minipage}{6.5cm}
						\centering
						球坐标系: \\
						{$fu_{g}=-\frac{1}{\rho{a}}\frac{\partial p}{\partial \theta}$\enspace
							$fv_{g}= \frac{1}{a\rho\cos{\theta}}\frac{\partial p}{\partial \lambda}$}
					\end{minipage} ($u_g,\enspace v_{g}$)为地转速度
				
				
				\end{itemize}
				
				
			\end{itemize}
		
	\section{无粘浅水理论}
	\section{行星边界层}
	\section{层结流体中的准地转运动}


	
\end{document}


